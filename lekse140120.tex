\documentclass[12pt]{article}
\usepackage[utf8]{inputenc}
\renewcommand{\baselinestretch}{1.5}
\usepackage[norsk]{babel}
\usepackage[a4paper, total={6in, 8in}]{geometry}
\usepackage{hyperref}

\usepackage{titlesec}

\titleformat*{\section}{\large\bfseries}
\titleformat*{\subsection}{\normal\bfseries}
\titleformat*{\subsubsection}{\small\bfseries}
\titleformat*{\paragraph}{\large\bfseries}
\titleformat*{\subparagraph}{\large\bfseries}

\usepackage{csquotes}

\title{Lekse 14.01}
\author{Arne Skjævesland}
\date{January 2020}

\begin{document}

\maketitle

\section{Tema og problemstillinger til min masteroppgave}
Temaet for min masteroppgave er kampen for de homofiles rettigheter med fokus på debatten rundt partnerskapsloven og 
ekteskapsloven. Jeg kommer til å se på hvilke holdninger som fantes/finnes mot homofile, men også hvordan homofile 
definerer seg selv, og deres synspunkter rundt likestillingsdebatten. Med ekteskapsloven kom også foreldrerettigheter 
hvor det blir like rettigheter i fht. adopsjon og assistert befruktning (for lesbiske  kvinner),\footnote{\url{https://lovdata.no/dokument/LTI/lov/2008-06-27-53}} slik at fokuset for denne oppgaven går 
ikke bare på retten til ekteskap, men også retten til familieliv.

Problemstillinger for oppgaven er som følger: \textquote{Hva slags holdninger fantes mot homofile i Norge rundt 
innføring av partnerskapsloven i 1993, og ekteskapsloven i 2008?}, og \textquote{hvordan ser homofile på seg selv, 
og hva slags meninger har de rundt rettigheter knyttet til ekteskap og familieliv?}

\section{Trolldomsprosessenes opphør i Skandinavia av Gunnar W. Knutsen}

\subsection{Tema, problemstillinger og tese}
\underline{Tema:} trolldomsprosessene i Skandinavia - dets nedgang og opphør. \\
\underline{Problemstillinger:} 1) tar for seg utviklingen av trolldomsprosessene i Skandinavia, og 2) forskjellige 
teorier som forklarer trolldomsprosessenes slutt. \\
\underline{Tese:} lite forskning på nedgang og opphør på trolldomsprosessene pga. antakelse om opplysning, økt 
kunnskap, rasjonalitet og mindre overtro i Europa på 1700-tallet som forklaring.

\subsection{Forskningssituasjon}
\textquote{De europeiske trolldomsprosessenes nedgang og opphør er et tema som har vært gjenstand for lite forskning, 
sammenliknet med deres begynnelse.}

\textquote{Empirien er mangelfull, men det er dog kommet mer kunnskap om sene trolldomsprosesser som ikke tidligere 
er brukt for å studere selve fenomenets nedgang og slutt. Teoriene som brukes for å forklare at trolldomsprosessene 
tok slutt, er derimot de samme som ble brukt for 25 år siden.} 

I tillegg sier han at de teoriene som har blitt brukt, men også de som fortsatt blir brukt, ikke er tilstrekkelige 
til å forklare nedgangen og opphøret av trolldomsprosessene. Han peker også på at selv \textquote{de nyeste 
standardtekstene om europeiske trolldomsprosesser allerede er utdaterte eller i det minste unøyaktige i sin omtale 
av Skandinavia.}.

\subsection{Teoretiske begreper}
Trolldomsprosess - brudd på trolldomslovgivning. Trolldom som et samlebegrep for overnaturlige forbrytelser.

Også hekseprosess (og hekseri) - underfenomen (gjaldt mer handlinger med et ondt formål, i motsetning til f.eks. 
helbredelse).

\subsection{Kilde og metoder}
Nevner ikke konkret hva slags kilder han skal ta for seg innledningsvis, bort sett fra et oversiktsverk, og snakker 
litt om kildesituasjonen med å fortelle at det ikke finnes oversiktsverk om trolldom i Danmark eller Sverige til 
tross for at det finnes mye forskning på trolldom, spesielt i Sverige.

Han skal blant annet se på antall trolldomssaker - hvor han nevner oversiktsverk som kilde (andre norske kilder: 
domsprotokoller og regnskap (lens- og fogderegnskap))\\
Teorier som er tatt i bruk i tidligere studier. \\
Problem: fraværet av nasjonale oversikter over trolldomsprosesser i Sverige og Danmark.

(Dette en kort artikkel, og ikke en masteroppgave)

\end{document}
